\section{Parameters}
\label{sec:cli:parameters}


The different command line interfaces, or CLIs, (\texttt{1-fuse.py}, \texttt{2-mars.py}, etc.) requires a parameter file (which is nothing but a \texttt{python} file) that contains both information on the experiment (path to the experiment directory, on the sub-directory names -- see section \ref{sec:cli:parameters:data}) as well as specific parameters for the CLIs.

\subsection{Prefixed parameters}

Some of the parameter sets are said to be \textit{prefixed}, such as the two sets of pre-processing parameters for the \texttt{2-mars.py} CLI (see section \ref{sec:cli:parameters:mars}). Indeed, the pre-processing can be set differently for the seed input image and the membrane input image (eg see section \ref{sec:cli:mars}).

Prefixing parameters allows to either set \textit{all} the parameters with the same name together or set them \textit{independently}.

As exemplified in section \ref{sec:cli:mars:input:images}, 
the parameter file lines (where the variables are not prefixed)
\begin{verbatim}
intensity_transformation = 'normalization_to_u8'
intensity_enhancement = None
\end{verbatim}
will set the corresponding pre-processing parameters for both the seed and the membrane image pre-processing.
However, using prefixes, as in the lines
\begin{verbatim}
seed_intensity_transformation = 'Identity'
membrane_intensity_transformation = 'normalization_to_u8'
intensity_enhancement = None
\end{verbatim}
allows to set them independently.

This mechanism is designed to simplify the parameter file, but may have undesired consequences. Indeed, using the basic variable names of the registration parameters (see section \ref{sec:cli:parameters:registration})  for the \texttt{4-astec.py} CLI will change all registration parameters included in the pre-processing parameters.

To check whether the parameters have been set correctly, one can either use
the \texttt{--print-param} CLI option (see section \ref{sec:cli:common}) beforehand, or to a posteriori check the used parameter in the \texttt{log} file. 
  
  

\subsection{Common parameters}
\label{sec:cli:parameters:common}

\begin{itemize}
\itemsep -0.5ex
\item \texttt{begin}:
  first time point to be processed (\texttt{1-fuse.py}, 
  \texttt{4-astec.py} or \texttt{5-postcorrection.py}) 
  or single time point to be processed
  (\texttt{2-mars.py} or \texttt{3-manualcorrection.py}).
\item \texttt{end}:
  last time point to be processed (\texttt{1-fuse.py}, 
  \texttt{4-astec.py} or \texttt{5-postcorrection.py}).
\item \texttt{delta}:
  interval between two time points to be processed. Set to 1 by default.
  Fragile.
\item \texttt{raw\_delay}:
  Delay to be added to the time points to build the file names. 
  Fragile.
\item \texttt{time\_digits\_for\_filename}:
  number of digits used to build the file names.
\item \texttt{time\_digits\_for\_cell\_id}:
  number of digits used to define unique cellule id. in lineage file.
  The unique id of cell $c$ at time $t$ is $t \times 10^d + c$ where
  $d$ is set by \texttt{time\_digits\_for\_cell\_id}.
\item \texttt{default\_image\_suffix}:
  used for both the result and the temporary data.
  \begin{itemize}
  \item \texttt{'inr'}: Inrimage format, historical choice.
  \item \texttt{'mha'}: MetaImage format, readable by Fiji.
  \item \texttt{'tif'}: not advised, since the tiff format does not allow
  to keep the voxel size along the z direction (at least in a 
  standardized way).
  \end{itemize}
\item \texttt{result\_image\_suffix}:
  used for both the result data.
\item \texttt{result\_lineage\_suffix}:
  \begin{itemize}
  \item \texttt{'pkl'}: 
  \item \texttt{'xml'}: 
  \end{itemize}
\end{itemize}


\subsection{Data organisation parameters}
\label{sec:cli:parameters:data}

\begin{figure}
\mbox{}
\dirtree{%
.1 \texttt{<PATH\_EMBRYO>}/.
.2 \texttt{<DIR\_RAWDATA>}/.
.3 \texttt{<DIR\_LEFTCAM\_STACKZERO>}/.
.4 \texttt{<acquisition\_leftcam\_image\_prefix>000.zip}.
.4 \texttt{<acquisition\_leftcam\_image\_prefix>001.zip}.
.4 \ldots.
.3 \texttt{<DIR\_RIGHTCAM\_STACKZERO>}/.
.4 \texttt{<acquisition\_rightcam\_image\_prefix>000.zip}.
.4 \texttt{<acquisition\_rightcam\_image\_prefix>001.zip}.
.4 \ldots.
.3 \texttt{<DIR\_LEFTCAM\_STACKONE>}/.
.4 \texttt{<acquisition\_leftcam\_image\_prefix>000.zip}.
.4 \texttt{<acquisition\_leftcam\_image\_prefix>001.zip}.
.4 \ldots.
.3 \texttt{<DIR\_RIGHTCAM\_STACKONE>}/.
.4 \texttt{<acquisition\_rightcam\_image\_prefix>000.zip}.
.4 \texttt{<acquisition\_rightcam\_image\_prefix>001.zip}.
.4 \ldots.
.2 \ldots.
}
\mbox{}
\caption{\label{fig:data:rawdata:1} Typical organisation of mono-channel data. See also section \protect\ref{sec:cli:fuse:input:data}.}
\end{figure}

\begin{figure}
\mbox{}
\dirtree{%
.1 \texttt{<PATH\_EMBRYO>}/.
.2 \texttt{<DIR\_RAWDATA>}/.
.3 \texttt{<DIR\_LEFTCAM\_STACKZERO>}/.
.4 \texttt{<acquisition\_leftcam\_image\_prefix>000.zip}.
.4 \ldots.
.3 \texttt{<DIR\_RIGHTCAM\_STACKZERO>}/.
.4 \texttt{<acquisition\_rightcam\_image\_prefix>000.zip}.
.4 \ldots.
.3 \texttt{<DIR\_LEFTCAM\_STACKONE>}/.
.4 \texttt{<acquisition\_leftcam\_image\_prefix>000.zip}.
.4 \ldots.
.3 \texttt{<DIR\_RIGHTCAM\_STACKONE>}/.
.4 \texttt{<acquisition\_rightcam\_image\_prefix>000.zip}.
.4 \ldots.
.3 \texttt{<DIR\_LEFTCAM\_STACKZERO\_CHANNEL\_2>}/.
.4 \texttt{<acquisition\_leftcam\_image\_prefix>000.zip}.
.4 \ldots.
.3 \texttt{<DIR\_RIGHTCAM\_STACKZERO\_CHANNEL\_2>}/.
.4 \texttt{<acquisition\_rightcam\_image\_prefix>000.zip}.
.4 \ldots.
.3 \texttt{<DIR\_LEFTCAM\_STACKONE\_CHANNEL\_2>}/.
.4 \texttt{<acquisition\_leftcam\_image\_prefix>000.zip}.
.4 \ldots.
.3 \texttt{<DIR\_RIGHTCAM\_STACKONE\_CHANNEL\_2>}/.
.4 \texttt{<acquisition\_rightcam\_image\_prefix>000.zip}.
.4 \ldots.
.2 \ldots.
}
\mbox{}
\caption{\label{fig:data:rawdata:2} Typical organisation of multi-channel data. See also section \protect\ref{sec:cli:fuse:input:data}.}
\end{figure}

\begin{figure}
\mbox{}
\dirtree{%
.1 \texttt{<PATH\_EMBRYO>}/.
.2 \texttt{<DIR\_RAWDATA>}/.
.3 \texttt{<DIR\_LEFTCAM\_STACKZERO>}/.
.4 \texttt{<acquisition\_leftcam\_image\_prefix>000.zip}.
.4 \ldots.
.3 \texttt{<DIR\_RIGHTCAM\_STACKZERO>}/.
.4 \texttt{<acquisition\_rightcam\_image\_prefix>000.zip}.
.4 \ldots.
.3 \texttt{<DIR\_LEFTCAM\_STACKONE>}/.
.4 \texttt{<acquisition\_leftcam\_image\_prefix>000.zip}.
.4 \ldots.
.3 \texttt{<DIR\_RIGHTCAM\_STACKONE>}/.
.4 \texttt{<acquisition\_rightcam\_image\_prefix>000.zip}.
.4 \ldots.
.2 \texttt{<DIR\_RAWDATA\_CHANNEL\_2>}/.
.3 \texttt{<DIR\_LEFTCAM\_STACKZERO>}/.
.4 \texttt{<acquisition\_leftcam\_image\_prefix>000.zip}.
.4 \ldots.
.3 \texttt{<DIR\_RIGHTCAM\_STACKZERO>}/.
.4 \texttt{<acquisition\_rightcam\_image\_prefix>000.zip}.
.4 \ldots.
.3 \texttt{<DIR\_LEFTCAM\_STACKONE>}/.
.4 \texttt{<acquisition\_leftcam\_image\_prefix>000.zip}.
.4 \ldots.
.3 \texttt{<DIR\_RIGHTCAM\_STACKONE>}/.
.4 \texttt{<acquisition\_rightcam\_image\_prefix>000.zip}.
.4 \ldots.
.2 \ldots.
}
\mbox{}
\caption{\label{fig:data:rawdata:3} Alternative organisation of multi-channel data. See also section \protect\ref{sec:cli:fuse:input:data}.}
\end{figure}

\begin{figure}
\mbox{}
\dirtree{%
.1 \texttt{<PATH\_EMBRYO>}/.
.2 \texttt{<DIR\_RAWDATA>}/.
.3 \ldots.
.2 \texttt{<FUSE>}/.
.3 \texttt{FUSE\_<EXP\_FUSE>}/.
.4 \texttt{<EN>\_fuse\_t000.<result\_image\_suffix>}.
.4 \texttt{<EN>\_fuse\_t001.<result\_image\_suffix>}.
.4 \ldots.
.2 \ldots.
}
\mbox{}
\caption{\label{fig:data:fusion} Typical organisation of fused images. See also section \protect\ref{sec:cli:fuse:output:data}.}
\end{figure}

\begin{itemize}
\itemsep -0.5ex
\item \texttt{DIR\_LEFTCAM\_STACKONE} 
see section \ref{sec:cli:fuse:input:data}, 
see figures \ref{fig:data:rawdata:1}, 
\ref{fig:data:rawdata:2}, 
and \ref{fig:data:rawdata:3}.
\item \texttt{DIR\_LEFTCAM\_STACKONE\_CHANNEL\_2} 
see section \ref{sec:cli:fuse:input:data}
\item \texttt{DIR\_LEFTCAM\_STACKONE\_CHANNEL\_3} 
see section \ref{sec:cli:fuse:input:data}
\item \texttt{DIR\_LEFTCAM\_STACKZERO} 
see section \ref{sec:cli:fuse:input:data},
see figures \ref{fig:data:rawdata:1}, 
\ref{fig:data:rawdata:2}, 
and \ref{fig:data:rawdata:3}.
\item \texttt{DIR\_LEFTCAM\_STACKZERO\_CHANNEL\_2} 
see section \ref{sec:cli:fuse:input:data}
\item \texttt{DIR\_LEFTCAM\_STACKZERO\_CHANNEL\_3} 
see section \ref{sec:cli:fuse:input:data}
\item \texttt{DIR\_RAWDATA} 
see section \ref{sec:cli:fuse:input:data},
see figures \ref{fig:data:rawdata:1}, 
\ref{fig:data:rawdata:2}, 
and \ref{fig:data:rawdata:3}.
\item \texttt{DIR\_RAWDATA\_CHANNEL\_2} 
see section \ref{sec:cli:fuse:input:data}
\item \texttt{DIR\_RAWDATA\_CHANNEL\_3} 
see section \ref{sec:cli:fuse:input:data}
\item \texttt{DIR\_RIGHTCAM\_STACKONE} 
see section \ref{sec:cli:fuse:input:data},
see figures \ref{fig:data:rawdata:1}, 
\ref{fig:data:rawdata:2}, 
and \ref{fig:data:rawdata:3}.
\item \texttt{DIR\_RIGHTCAM\_STACKONE\_CHANNEL\_2} 
see section \ref{sec:cli:fuse:input:data}
\item \texttt{DIR\_RIGHTCAM\_STACKONE\_CHANNEL\_3} 
see section \ref{sec:cli:fuse:input:data}
\item \texttt{DIR\_RIGHTCAM\_STACKZERO} 
see section \ref{sec:cli:fuse:input:data},
see figures \ref{fig:data:rawdata:1}, 
\ref{fig:data:rawdata:2}, 
and \ref{fig:data:rawdata:3}.
\item \texttt{DIR\_RIGHTCAM\_STACKZERO\_CHANNEL\_2} 
see section \ref{sec:cli:fuse:input:data}
\item \texttt{DIR\_RIGHTCAM\_STACKZERO\_CHANNEL\_3} 
see section \ref{sec:cli:fuse:input:data}
\item \texttt{EN}: 
  the so=called \textit{embryo} name. 
  All files will be named after this name.
  E.g. see section \ref{sec:cli:fuse:output:data}.
see section \ref{sec:cli:fuse:output:data},
see figure \ref{fig:data:fusion}.
\item \texttt{EXP\_FUSE}:
  String (\texttt{str} type) or list (\texttt{list} type) of
  strings. It indicates what are the fused images directories, 
  of the form \texttt{<PATH\_EMBRYO>/FUSE/FUSE\_<EXP\_FUSE>}.
\begin{verbatim}
EXP_FUSE = 'exp1'
EXP_FUSE = ['exp1', 'exp2']
\end{verbatim}
  are then both valid. 
  Default value of \texttt{EXP\_FUSE} is \texttt{'RELEASE'}.
  See section \ref{sec:cli:fuse:output:data},
  see figure \ref{fig:data:fusion}.
\item \texttt{EXP\_FUSE\_CHANNEL\_2} 
see section \ref{sec:cli:fuse:output:data}
\item \texttt{EXP\_FUSE\_CHANNEL\_3} 
see section \ref{sec:cli:fuse:output:data}
\item \texttt{PATH\_EMBRYO}: 
  path to the \textit{experiment}.
  If not present, the current directory is used.
  See section \ref{sec:cli:fuse:input:data},
  see figures \ref{fig:data:rawdata:1}, 
\ref{fig:data:rawdata:2}, 
\ref{fig:data:rawdata:3}, and
\ref{fig:data:fusion}
\item \texttt{acquisition\_leftcam\_image\_prefix}  
see section \ref{sec:cli:fuse:input:data},
see figures \ref{fig:data:rawdata:1}, 
\ref{fig:data:rawdata:2}, 
and \ref{fig:data:rawdata:3}.
\item \texttt{acquisition\_rightcam\_image\_prefix}  
see section \ref{sec:cli:fuse:input:data},
see figures \ref{fig:data:rawdata:1}, 
\ref{fig:data:rawdata:2}, 
and \ref{fig:data:rawdata:3}.
\end{itemize}














\subsection{Ace parameters}
\label{sec:cli:parameters:ace}

Ace stand for \textit{Automated Cell Extractor}. [G[L]]ACE methods aim at detecting and enhancing membranes in a 3D images (see also section \ref{sec:cli:input:image:preprocessing:membrane}).
\begin{enumerate}
\itemsep -0.5ex
\item \label{it:ace:detection} Hessian-based detection of 2-D manifolds, computation of a center-membrane image.
\item \label{it:ace:thresholding} Thresholding of the center-membrane image to get a binary image.
\item \label{it:ace:tensor} Reconstruction of a membrane images from the binary image through tensor voting.
\end{enumerate}

\begin{itemize}
\itemsep -0.5ex
\item \texttt{sigma\_membrane}:
  this is the gaussian sigma that is used to compute image derivatives
  (in real units), used in step \ref{it:ace:detection}.
\item \texttt{hard\_thresholding}:
  \texttt{True} or \texttt{False}.
  If set to \texttt{True}, a hard threshold (set by variable 
  \texttt{hard\_threshold}) is used instead of an automated threshold.
\item \texttt{hard\_threshold}
\item \texttt{manual}: \texttt{True} or \texttt{False}.
  By default, this parameter is set to False. If failure, 
  (meaning that thresholds are very bad, meaning that the binarized 
  image is very bad), set this parameter to True and relaunch the
  computation on the test image. If the method fails again, "play" 
  with the value of \texttt{manual\_sigma} ... and good luck.
\item \texttt{manual\_sigma}:
  Axial histograms fitting initialization parameter for the computation 
  of membrane image binarization axial thresholds (this parameter is 
  used if \texttt{manual} is set to \texttt{True}). 
  One may need to test different values of 
  \texttt{manual\_sigma}. We suggest to test values between 5 and 25 
  in case of initial failure. Good luck.
\item \texttt{sensitivity}: membrane binarization parameter.
  Use larger values (smaller than or equal to 1.0) to increase 
  the quantity of binarized membranes to be used for tensor voting.
\item \texttt{sigma\_TV}: 
  parameter which defines the voting scale for membrane structures 
  propagation by tensor voting method (real coordinates). This parameter
  should be set between \SI{3}{\micro m} (little cells) and 
  \SI{4.5}{\micro m} (big gaps in 
  the binarized membrane image).
\item \texttt{sigma\_LF}:
  Additional smoothing parameter for reconstructed image 
  (in real coordinates).
  It seems that the default value = \SI{0.9}{\micro m} is 
  ok for standard use.
\item \texttt{sample}:
  Set the fraction of the binarized membranes (obtained at step  
  \ref{it:ace:thresholding}) further used for tensor voting.
  It allows tensor voting computation speed optimisation (do not 
  touch if not bewared): the more sample, the higher the cost.
\item \texttt{sample\_random\_seed}:
  Drawing a sample from the binarized membranes (see parameter 
  \texttt{sample}) is a stochastic process. Setting this parameter 
  to some \texttt{int} value allows to make this stochastic process 
  reproducible.
\item \texttt{bounding\_box\_dilation}
\item \texttt{default\_image\_suffix}
\end{itemize}










\subsection{Morphosnake parameters}
\label{sec:cli:parameters:morphosnake}

\begin{itemize}
\itemsep -0.5ex
\item \texttt{dilation\_iterations}: 
  dilation of the cell bounding box for computation purpose.
\item \texttt{iterations}:
  maximal number of morphosnake iterations.
\item \texttt{delta\_voxel}: 
  error on voxel count to define a stopping criteria.
\item \texttt{energy}:
  \begin{itemize}
  \item \texttt{'gradient'}: uses the same formula as in 
  \cite{marquez-neil:pami:2014}, as in the historical 
  astec version. But seems to be a poor choice.
  \item \texttt{'image'}: uses directly the image as the energy image.
  \end{itemize}
\item \texttt{smoothing}:
  internal parameter for the morphosnake.
\item \texttt{balloon}:
  internal parameter for the morphosnake.
\item \texttt{processors}: number of processors used for the 
  morphosnake correction.
\item \texttt{mimic\_historical\_astec}:
  \texttt{True} or \texttt{False}. 
  If set to \texttt{True}, same implementation than the historical 
  astec version. Kept for comparison purpose.
\end{itemize}










\subsection{Preprocessing parameters}
\label{sec:cli:parameters:preprocessing}

The input image may be pre-processed before being used as
\begin{itemize}
\itemsep -0.5ex
\item either the \textit{membrane} image (ie the height image) for watershed segmentation,
\item or the \textit{seed} image (ie the image with which the regional minima are computed),
\item or the \textit{morphosnake} image (ie the image with which the morphosnake energy is computed).
\end{itemize}
For more details, see section \ref{sec:cli:input:image:preprocessing}.

\begin{itemize}
\itemsep -0.5ex
\item Ace parameters
  (see section \ref{sec:cli:parameters:ace})
  
\item \texttt{intensity\_transformation}:
set the (histogram based) intensity transformation of the original image
(see section \ref{sec:cli:input:image:preprocessing:histogram})
\begin{itemize}
\itemsep -0.5ex
\item \texttt{None}: no intensity transformation of the original image is used to pre-process the input image.
\item \texttt{'identity'}: the input image is used without any transformation.
\item \texttt{'normalization\_to\_u8'}: the input image (usually encoded on 16 bits) is normalized onto 8 bits. The values corresponding to percentiles given by the variables \texttt{normalization\_min\_percentile} and  \texttt{normalization\_max\_percentile} are mapped respectively on 0 and 255.
\item \texttt{'cell\_normalization\_to\_u8'}: same principle than \texttt{'normalization\_to\_u8'} but values mapped on 0 and 255 are computed on a cell basis (cells are the ones of $S^{\star}_{t-1} \circ \mathcal{T}_{t-1 \leftarrow t}$ -- see \cite{guignard:tel-01278725} for notations --, ie the segmentation obtained for the previous time point $t-1$ and deformed onto the frame at the current time point $t$). This can be used only with \texttt{4-astec.py}
(section \ref{sec:cli:astec}).
This feature has been added for tests, but has not demonstrated yet any benefit.
\end{itemize}

\item \texttt{intensity\_enhancement}
set the membrane enhancement transformation of the original image
(see section \ref{sec:cli:input:image:preprocessing:membrane})
\begin{itemize}
\itemsep -0.5ex
\item \texttt{None}: no membrane enhancement of the original image is used to pre-process the input image.
\item \texttt{'GACE'}:
stands for \textit{Global Automated Cell Extractor}. It tries to reconstructed a membrane image through a membrane detector, an automated thresholding and a tensor voting step. The automated thresholding is computed once for the whole image.
\item \texttt{'GLACE'}: stands for \textit{Grouped Local Automated Cell Extractor}. It differs from one step from \texttt{GACE}: the threshold of extrema image is not computed globally (as in \texttt{GACE}), but one threshold is computed per cell of $S^{\star}_{t-1} \circ \mathcal{T}_{t-1 \leftarrow t}$, from the extrema values of the cell bounding box.
This can be used only with \texttt{4-astec.py}
(section \ref{sec:cli:astec}).
\end{itemize}

\item \texttt{outer\_contour\_enhancement}

\item \texttt{reconstruction\_images\_combination}:
  \begin{itemize}
  \item \texttt{'addition'}
  \item \texttt{'maximum'}
  \end{itemize}
\item \texttt{cell\_normalization\_min\_method}:
  set the cell area where is computed the percentile value that 
  will give the $0$ value in the normalized image
  \begin{itemize}
  \item \texttt{'cell'}
  \item \texttt{'cellborder'}
  \item \texttt{'cellinterior'}
  \end{itemize}
\item \texttt{cell\_normalization\_max\_method}:
  set the cell area where is computed the percentile value that 
  will give the $255$ value in the normalized image
  \begin{itemize}
  \item \texttt{'cell'}
  \item \texttt{'cellborder'}
  \item \texttt{'cellinterior'}
  \end{itemize}
\item \texttt{normalization\_min\_percentile}
\item \texttt{normalization\_max\_percentile}
\item \texttt{cell\_normalization\_sigma}:
the \texttt{'cell\_normalization\_to\_u8'} method computes a couple $(I_{min}, I_{max})$ for each cell of $S^{\star}_{t-1} \circ \mathcal{T}_{t-1 \leftarrow t}$, yielding discontinuities in the $I_{min}$ and $I_{max}$ from cell to cell. To normalize the whole image, images of $I_{min}$ and $I_{max}$ are built and then smoothed with a gaussian kernel (sigma given by the variable \texttt{cell\_normalization\_sigma}.
\item \texttt{intensity\_transformation}
\item Registration parameters 
  (see section \ref{sec:cli:parameters:registration}) prefixed 
  by \texttt{linear\_registration\_}
\item Registration parameters 
  (see section \ref{sec:cli:parameters:registration}) prefixed 
  by \texttt{nonlinear\_registration\_}
\item \texttt{keep\_reconstruction}:
  \texttt{True} or \texttt{False}. Is set to \texttt{True}, 
  pre-processed images are kept in a \texttt{RECONSTRUCTION/} directory.
\end{itemize}










\subsection{Registration parameters}
\label{sec:cli:parameters:registration}

\begin{itemize}
\itemsep -0.5ex
\item \texttt{compute\_registration}
\item \texttt{pyramid\_highest\_level}:
  highest level of the pyramid image for registration.
  Registration is done hierarchically with a pyramid of images. At 
  each pyramid level, image dimensions are divided by 2.
  Setting this variable to 6 means that registration starts with images 
  whose dimensions are 1/64th of the original image.
\item \texttt{pyramid\_lowest\_level}:
  lowest level of the pyramid image for registration. Setting it
  to 0 means that the lowest level is with the image itself.
  Setting it to 1 or even 2 allows to gain computational time.
\item \texttt{gaussian\_pyramid}
\item \texttt{transformation\_type}
\item \texttt{elastic\_sigma}
\item \texttt{transformation\_estimation\_type}
\item \texttt{lts\_fraction}
\item \texttt{fluid\_sigma}
\item \texttt{normalization}
\end{itemize}










\subsection{Seed edition parameters}
\label{sec:cli:parameters:seed:edition}

\begin{itemize}
\itemsep -0.5ex
\item \texttt{seed\_edition\_dir}:
\item \texttt{seed\_edition\_file}:
  if run with \texttt{'-k'}, temporary files, including the computed 
  seeds are kept into a temporary directory, and can be corrected in
  several rounds
\begin{verbatim}
seed_edition_file = [['seeds_to_be_fused_001.txt', 'seeds_to_be_created_001.txt'], \
                     ['seeds_to_be_fused_002.txt', 'seeds_to_be_created_002.txt'],
                     ...
                     ['seeds_to_be_fused_00X.txt', 'seeds_to_be_created_00X.txt']]
\end{verbatim}
  Each line of a \texttt{seeds\_to\_be\_fused\_00x.txt} file contains 
  the labels to be fused, e.g. "10 4 2 24". A same label can be found 
  in several lines, meaning that all the labels of these lines will be 
  fused. Each line of \texttt{seeds\_to\_be\_created\_00x.txt} contains 
  the coordinates of a seed to be added.
\end{itemize}


\subsection{Watershed parameters}
\label{sec:cli:parameters:watershed}

\begin{itemize}
\itemsep -0.5ex
\item \texttt{seed\_sigma}:
  gaussian sigma for smoothing of initial image for seed extraction
  (real coordinates).
\item \texttt{seed\_hmin}:
  $h$ value for the extraction of the $h$-minima,
\item \texttt{seed\_high\_threshold}:
  regional minima thresholding. 
\item \texttt{membrane\_sigma}:
  gaussiab sigma for smoothing of reconstructed image for image 
  regularization prior to segmentation
  (real coordinates).
\end{itemize}










\subsection{\texttt{1-fuse.py} parameters}
\label{sec:cli:parameters:fuse}


\begin{itemize}
\itemsep -0.5ex
\item \texttt{acquisition\_orientation}:
  image orientation (\texttt{'right'} or \texttt{'left'})
  gives the rotation (with respect to the Y axis) of the left camera 
  frame of stack \#0 to be aligned with the the left camera 
  frame of stack \#1.
  \begin{itemize}
  \itemsep -0.5ex
  \item \texttt{'right'}: +90 degrees
  \item \texttt{'left'}: -90 degrees
  \end{itemize}
  See section \ref{sec:cli:fuse:important:parameters}.
\item \texttt{acquisition\_mirrors}:
  mirroring of the right camera image along the X-axis.
  Right camera images may have to be mirrored along the X-axis 
  to be aligned with the left camera images.
  \begin{itemize}
  \itemsep -0.5ex
  \item \texttt{True}: +90 degrees
  \item \texttt{False}: -90 degrees
  \end{itemize}
  Since it should depend on the apparatus,
  this parameter should not change for all acquisitions 
  performed by the same microscope.
  See section \ref{sec:cli:fuse:important:parameters}.
\item \texttt{acquisition\_resolution}:
  acquisition voxel size
  e.g. 
  \begin{verbatim}
  raw_resolution = (.21, .21, 1.)
  \end{verbatim}
  see section \ref{sec:cli:fuse:important:parameters}
\item \texttt{acquisition\_stack0\_leftcamera\_z\_stacking}:
  see \texttt{acquisition\_leftcamera\_z\_stacking}.
\item \texttt{acquisition\_stack1\_leftcamera\_z\_stacking}:
  see \texttt{acquisition\_leftcamera\_z\_stacking}.
\item \texttt{acquisition\_slit\_line\_correction}:
  \texttt{True} or \texttt{False}.
  See section \ref{sec:cli:fuse:overview}.
\item \texttt{target\_resolution}:
  isotropic voxel size of the fusion result (fused images).
  See section \ref{sec:cli:fuse:output:data}.
  
\item \texttt{fusion\_strategy}:
  \begin{itemize}
  \itemsep -0.5ex
  \item \texttt{'direct-fusion'}: each acquisition is linearly 
  co-registered with the first acquisition (stack \#0, left camera). 
  Used registration parameters are the ones  prefixed by 
  \texttt{fusion\_preregistration\_} and 
  \texttt{fusion\_registration\_}.
  Then weights and images are transformed thanks to the 
  computed transformations.
  \item \texttt{'hierarchical-fusion'}: from the couple 
  (left camera, right camera), each stack is reconstructed (with the
  registration parameters prefixed by 
  \texttt{fusion\_preregistration\_} and 
  \texttt{fusion\_registration\_}), following the same scheme than 
  the direct fusion but with only 2 images. 
  Then stack\#1 is (non-)linearly co-registered with stack \#0 with the
  registration parameters prefixed by 
  \texttt{fusion\_stack\_preregistration\_} and 
  \texttt{fusion\_stack\_registration\_}.
   Images and weights associated with stack\#1 are then (non-)linearly 
   transformed. Finally a weighted linear combination gives the result.
  \end{itemize}
  See section \ref{sec:cli:fuse:image:coregistration}
  
\item \texttt{acquisition\_cropping}:
  \texttt{True} or \texttt{False}. If set to \texttt{True}, 
  the acquisitions stacks are cropped before fusion.
  See section \ref{sec:cli:fuse:raw:data:cropping}
\item \texttt{acquisition\_cropping\_margin\_x\_0}:
  extra margin for the left side of the X direction.
\item \texttt{acquisition\_cropping\_margin\_x\_1}:
  extra margin for the right side of the X direction.
\item \texttt{acquisition\_cropping\_margin\_y\_0}:
  extra margin for the left side of the Y direction.
\item \texttt{acquisition\_cropping\_margin\_y\_1}:
  extra margin for the right side of the Y direction.
\item \texttt{acquisition\_cropping\_margin\_x}: 
  allows to set both \texttt{acquisition\_cropping\_margin\_x\_0} and
  \texttt{acquisition\_cropping\_margin\_x\_1}
\item \texttt{acquisition\_cropping\_margin\_y}: 
  allows to set both \texttt{acquisition\_cropping\_margin\_y\_0} and
  \texttt{acquisition\_cropping\_margin\_y\_1}
\item \texttt{acquisition\_cropping\_margin}: 
  allows to set the four margin variables.
  
\item Registration parameters 
  (see section \ref{sec:cli:parameters:registration}) prefixed 
  by \texttt{fusion\_preregistration\_}
\item Registration parameters 
  (see section \ref{sec:cli:parameters:registration}) prefixed 
  by \texttt{fusion\_registration\_}
\item Registration parameters 
  (see section \ref{sec:cli:parameters:registration}) prefixed 
  by \texttt{fusion\_stack\_preregistration\_}
\item Registration parameters 
  (see section \ref{sec:cli:parameters:registration}) prefixed 
  by \texttt{fusion\_stack\_registration\_}
  
\item \texttt{xzsection\_extraction}:
  \texttt{True} or \texttt{False}.
  Setting \texttt{xzsection\_extraction} to \texttt{True} allows 
  to extract XZ-sections of the 4 co-registered stacks as well as 
  the weighting function images. It provides an efficient way to check
  whether the \texttt{acquisition\_leftcamera\_z\_stacking} variable 
  was correcly set.
  See section \ref{sec:cli:fuse:stack:fusion}
  
\item \texttt{fusion\_cropping}:
  \texttt{True} or \texttt{False}. If set to \texttt{True}, 
  the fusion result is cropped.
  see section \ref{sec:cli:fuse:fused:data:cropping}
\item \texttt{fusion\_cropping\_margin\_x\_0}
\item \texttt{fusion\_cropping\_margin\_x\_1}
\item \texttt{fusion\_cropping\_margin\_y\_0}
\item \texttt{fusion\_cropping\_margin\_y\_1}
\item \texttt{fusion\_cropping\_margin\_x}:
  allows to set both \texttt{fusion\_cropping\_margin\_x\_0}
  and \texttt{fusion\_cropping\_margin\_x\_1}
\item \texttt{fusion\_cropping\_margin\_y}:
  allows to set both \texttt{fusion\_cropping\_margin\_y\_0}
  and \texttt{fusion\_cropping\_margin\_y\_1}
\item \texttt{fusion\_cropping\_margin}:
  allows to set the four margin variables.
  
\item \texttt{acquisition\_leftcamera\_z\_stacking}:
  allows to set both \texttt{acquisition\_stack0\_leftcamera\_z\_stacking} 
  and \texttt{acquisition\_stack1\_leftcamera\_z\_stacking}.
  Gives the order of stacking of in the Z direction
  \begin{itemize}
  \itemsep -0.5ex
  \item \texttt{'direct'}: from the high-contrasted images (small values of z) to the fuzzy/blurred ones (large values of z)
  \item \texttt{'inverse'}: the other way around.
  \end{itemize}
  See section \ref{sec:cli:fuse:important:parameters}.

\item \texttt{fusion\_weighting}: 
  set the weighting function for the weighted sum of the registered
  acquisition stacks (for all channels to be processed).
  \begin{itemize}
  \itemsep -0.5ex
  \item \texttt{'uniform'}: uniform (or constant) weighting, it comes 
  to the average of the resampled co-registered stacks
  \item \texttt{'ramp'}: the weights are linearly increasing or 
  decreasing along the Z axis
  \item \texttt{'corner'}: the weights are constant in a corner portion 
  of the stack, defined by two diagonals in the XZ-section
  \item \texttt{'guignard'}: original historical weighting function, 
  described in Leo Guignard's Phd thesis \cite{guignard:tel-01278725}, 
  that puts more weight to sections close to the camera and take
  also account the traversed material.
  \end{itemize}
  See section \ref{sec:cli:fuse:stack:fusion}.
\item \texttt{fusion\_weighting\_channel\_1}:
  set the weighting function for the weighted sum of the registered
  acquisition stacks for the first channel (in case of multi-channel
  acquisition).
\item \texttt{fusion\_weighting\_channel\_2}:
  set the weighting function for the weighted sum of the registered
  acquisition stacks for the second channel (in case of multi-channel
  acquisition).
\item \texttt{fusion\_weighting\_channel\_3}:
  set the weighting function for the weighted sum of the registered
  acquisition stacks for the third channel (in case of multi-channel
  acquisition).
\end{itemize}

The following parameters are kept for backward compatibility:
\begin{itemize}
\itemsep -0.5ex
\item \texttt{fusion\_crop} 
  same as \texttt{fusion\_cropping}
\item \texttt{fusion\_margin\_x\_0}
  same as \texttt{fusion\_cropping\_margin\_x\_0}
\item \texttt{fusion\_margin\_x\_1}
  same as \texttt{fusion\_cropping\_margin\_x\_1}
\item \texttt{fusion\_margin\_y\_0}
  same as \texttt{fusion\_cropping\_margin\_y\_0}
\item \texttt{fusion\_margin\_y\_1}
  same as \texttt{fusion\_cropping\_margin\_y\_1}
\item \texttt{fusion\_xzsection\_extraction} 
  same as \texttt{xzsection\_extraction}
\item \texttt{raw\_crop} 
  same as \texttt{acquisition\_cropping}
\item \texttt{raw\_margin\_x\_0}
  same as \texttt{acquisition\_cropping\_margin\_x\_0}
\item \texttt{raw\_margin\_x\_1}
  same as \texttt{acquisition\_cropping\_margin\_x\_1}
\item \texttt{raw\_margin\_y\_0}
  same as \texttt{acquisition\_cropping\_margin\_y\_0}
\item \texttt{raw\_margin\_y\_1}
  same as \texttt{acquisition\_cropping\_margin\_y\_1}
\item \texttt{raw\_mirrors} 
  same as \texttt{acquisition\_mirrors}
\item \texttt{raw\_ori} 
  same as \texttt{acquisition\_orientation}
\item \texttt{raw\_resolution} 
  same as \texttt{acquisition\_resolution}
\end{itemize}

\begin{itemize}
\itemsep -0.5ex
\item \texttt{begin} 
  see section \ref{sec:cli:fuse:important:parameters}
\item \texttt{delta}
\item \texttt{end} 
  see section \ref{sec:cli:fuse:important:parameters}
\item \texttt{fusion\_weighting}
\item \texttt{fusion\_weighting\_channel\_1}
\item \texttt{fusion\_weighting\_channel\_2}
\item \texttt{fusion\_weighting\_channel\_3}
\item \texttt{raw\_delay}
\end{itemize}










\subsection{\texttt{1.5-intraregistration.py} parameters}
\label{sec:cli:parameters:intraregistration}

These parameters are prefixed by \texttt{intra\_registration\_}.
\begin{itemize}
\itemsep -0.5ex
\item Registration parameters 
(see section \ref{sec:cli:parameters:registration}) 

\item \texttt{reference\_index}:
  defines the still image after transformation compositions it will 
  only translated, except if \texttt{reference\_transformation\_file} 
  or \texttt{reference\_transformation\_angles} are set.
  See section \ref{sec:cli:intraregistration:template}.
\item \texttt{reference\_transformation\_file}:
  resampling transformation to be applied to the reference image 
  (and to the whole serie) after transformation compositions.
  See section \ref{sec:cli:intraregistration:template}.
\item \texttt{reference\_transformation\_angles}:
  list of rotations wrt the X, Y,or Z axis that defines the resampling
  transformation.
\begin{verbatim}
reference_transformation_angles = 'X 30 Y 50'
\end{verbatim}
  represents a rotation of 30 degree around the X axis followed by a 
  rotation of 50 degrees around the Y axis.
  
  Beware: rotation composition depends on the order, so 
  \texttt{'X 30 Y 50'} is not equivalent to \texttt{'Y 50 X 30'}.

\item \texttt{template\_type}
\item \texttt{template\_threshold}
\item \texttt{margin}
\item \texttt{resolution}
\item \texttt{rebuild\_template}:
  \texttt{True} or \texttt{False}.
  If set to \texttt{True}, force to recompute the template as well as the
  transformations from existing co-registrations (that are not
  re-computed). It is useful when a first intra-registration has been
  done with only the fusion images: a second intra-registration with
  the segmentation images as template can be done without recomputing 
  the co-registrations.
\item \texttt{sigma\_segmentation\_images}
\item \texttt{resample\_fusion\_images}
\item \texttt{resample\_segmentation\_images}
\item \texttt{resample\_post\_segmentation\_images}
\item \texttt{movie\_fusion\_images}
\item \texttt{movie\_segmentation\_images}
\item \texttt{movie\_post\_segmentation\_images}
\item \texttt{xy\_movie\_fusion\_images}
\item \texttt{xz\_movie\_fusion\_images}
\item \texttt{yz\_movie\_fusion\_images}
\item \texttt{xy\_movie\_segmentation\_images}
\item \texttt{xz\_movie\_segmentation\_images}
\item \texttt{yz\_movie\_segmentation\_images}
\item \texttt{xy\_movie\_post\_segmentation\_images}
\item \texttt{xz\_movie\_post\_segmentation\_images}
\item \texttt{yz\_movie\_post\_segmentation\_images}
\item \texttt{maximum\_fusion\_images}
\item \texttt{maximum\_segmentation\_images}
\item \texttt{maximum\_post\_segmentation\_images}
\end{itemize}










\subsection{\texttt{2-mars.py} parameters}
\label{sec:cli:parameters:mars}

These parameters are prefixed by \texttt{mars\_}.
 
\begin{itemize}
\itemsep -0.5ex
\item \texttt{first\_time\_point}:
  first time point to be segmented by the mars method.
  Overrides the value of the \texttt{begin} variable.
\item \texttt{last\_time\_point}:
  last time point to be segmented by the mars method.
\item Watershed parameters 
  (see section \ref{sec:cli:parameters:watershed})
\item Seed edition parameters
  (see section \ref{sec:cli:parameters:seed:edition})
\item Preprocessing parameters
  (see section \ref{sec:cli:parameters:preprocessing})
  prefixed by \texttt{seed\_}
\item Preprocessing parameters
  (see section \ref{sec:cli:parameters:preprocessing})
  prefixed by \texttt{membrane\_}
\end{itemize}










\subsection{\texttt{3-manualcorrection.py} parameters}
\label{sec:cli:parameters:manualcorrection}

\begin{itemize}
\itemsep -0.5ex
\item \texttt{first\_time\_point}:
  first time point to be corrected.
  Overrides the value of the \texttt{begin} variable.
\item \texttt{last\_time\_point}:
  lats time point to be corrected.
\item \texttt{input\_image}:
  defines the input file names (to be used when correcting
  other files than the \texttt{2-mars.py} output file.
\item \texttt{output\_image}:
  defines the output file names (to be used when correcting
  other files than the \texttt{2-mars.py} output file.
\item \texttt{mapping\_file}:
  path to mapping file for manual correction of a segmentation (ie label)
  image. See above the syntax of this file.
  \begin{itemize}
  \itemsep -0.5ex
  \item 1 line per label association
  \item background label has value 1
  \item the character \texttt{\#} denotes commented lines 
  \end{itemize}
  Example of \texttt{mapping\_file}:
\begin{verbatim}
# here the input label 8 will be mapped with new value 7, etc...
8 7
9 2  
4 64 
29 23
# ... etc ...
# background labels
30 1 
89 1 
\end{verbatim}
\end{itemize}










\subsection{\texttt{4-astec.py} parameters}
\label{sec:cli:parameters:astec}


These parameters are prefixed by \texttt{astec\_}.
 
\begin{itemize}
\itemsep -0.5ex
\item Watershed parameters 
  (see section \ref{sec:cli:parameters:watershed})
\item Preprocessing parameters
  (see section \ref{sec:cli:parameters:preprocessing})
  prefixed by \texttt{seed\_}
\item Preprocessing parameters
  (see section \ref{sec:cli:parameters:preprocessing})
  prefixed by \texttt{membrane\_}
\item Preprocessing parameters
  (see section \ref{sec:cli:parameters:preprocessing})
  prefixed by \texttt{morphosnake\_}
\item Morphosnake parameters
  (see section \ref{sec:cli:parameters:morphosnake})
\item \texttt{propagation\_strategy}:
  \begin{itemize}
  \item \texttt{'seeds\_from\_previous\_segmentation'}
  \item \texttt{'seeds\_selection\_without\_correction'}
  \end{itemize}
\item \texttt{previous\_seg\_method}:
  how to build the seeds $S^e_{t-1 \leftarrow t}$ 
  for the computation of $\tilde{S}_{t}$
  \begin{itemize}
  \item \texttt{'deform\_then\_erode'}: $S^{\star}_{t-1}$ is transformed
  towards $I_t$ frame through $\mathcal{T}_{t-1 \leftarrow t}$, and then
  the cells and the background  are eroded.
  \item \texttt{'erode\_then\_deform'}: historical method. The cells 
  and the background of $S^{\star}_{t-1}$ are eroded, and then transformed
  towards $I_t$ frame through $\mathcal{T}_{t-1 \leftarrow t}$.
  \end{itemize}
\item \texttt{previous\_seg\_erosion\_cell\_iterations}:
  set the cell erosion size for $S^e_{t-1 \leftarrow t}$ computation. 
\item \texttt{previous\_seg\_erosion\_background\_iterations}:
  set the background erosion size for $S^e_{t-1 \leftarrow t}$ computation. 
\item \texttt{previous\_seg\_erosion\_cell\_min\_size}:
  size threshold. Cells whose size is below this threshold will 
  be discarded seeds in $S^e_{t-1 \leftarrow t}$ 

\item \texttt{watershed\_seed\_hmin\_min\_value}:
  set the $h_{min}$ value of the $[h_{min}, h_{max}]$ interval.
\item \texttt{watershed\_seed\_hmin\_max\_value}:
  set the $h_{max}$ value of the $[h_{min}, h_{max}]$ interval.
\item \texttt{watershed\_seed\_hmin\_delta\_value}
  set the $\delta h$ to go from one $h$ to the next in the 
  $[h_{min}, h_{max}]$ interval.
  
\item \texttt{background\_seed\_from\_hmin}:
  \texttt{True} or \texttt{False}. 
  Build the background seed at time point $t$ by cell propagation.  
\item \texttt{background\_seed\_from\_previous}:
  \texttt{True} or \texttt{False}.
  Build the background seed at time point $t$ by using the background 
  seed from $S^e_{t-1 \leftarrow t}$. 
  Fragile. 
  
\item \texttt{seed\_selection\_tau}:
  Set the $\tau$ value for division decision (seed selection step).
 
\item \texttt{minimum\_volume\_unseeded\_cell}:
  Volume threshold for cells without found seeds in the seed 
  selection step. Cells with volume (in $\tilde{S}_t$) whose size is below
  this threshold and for which no seed was found are discarded.

\item \texttt{volume\_ratio\_tolerance}:
  Ratio threshold to decide whether there is a volume decrease 
  (due to the background) for morphosnake correction.
\item \texttt{volume\_ratio\_threshold}:
  Ratio threshold to decide whether there is a large volume decrease
  for segmentation consistency checking.
\item \texttt{volume\_minimal\_value}:
  Size threshold for seed correction step. For a given cell at time 
  point $t-1$, if the corresponding cell(s) at time point $t$ has(ve) 
  volume below this threshold, they are discarded (and the cell at time 
  point $t-1$ has no lineage.
  
\item \texttt{morphosnake\_correction}:
  \texttt{True} or \texttt{False}. 
\item \texttt{outer\_correction\_radius\_opening}
\end{itemize}










\subsection{\texttt{5-postcorrection.py} parameters}
\label{sec:cli:parameters:postcorrection}

These parameters are prefixed by \texttt{postcorrection\_}.

\begin{itemize}
\itemsep -0.5ex
\item \texttt{volume\_minimal\_value}
  branch ending with leaf cell below this value are candidate for deletion. Expressed in voxel unit.
\item \texttt{lifespan\_minimal\_value}
\item \texttt{test\_early\_division}
\item \texttt{test\_volume\_correlation}
\item \texttt{correlation\_threshold}
\item \texttt{test\_postponing\_division}
\item \texttt{postponing\_correlation\_threshold}
\item \texttt{postponing\_minimal\_length}
\item \texttt{postponing\_window\_length}
\item \texttt{lineage\_diagnosis}
  performs a kind of diagnosis on the lineage before and after the post-correction.
\end{itemize}





