\section{\texttt{1.5-intraregistration.py}}
\label{sec:cli:intraregistration}

The sequence intra-registration procedure can be done either after the fusion step, or after the (post-)segmentation step. It aims at
\begin{itemize}
\itemsep -1ex
\item compensating for the eventual motion of the imaged sample with respect to the microscope
\item resampling the fusion and/or the segmentation images into a common frame/geometry, so they can better be compared, and
\item building 2D+t images made of 2D sections from either the  fusion
  and/or the segmentation images, so that the quality of the fusion
  and/of the tracking step can be visually assessed. 
\end{itemize}


The intra-registration procedure is made of the following steps:
\begin{enumerate}
\itemsep -1ex
\item \label{it:intrareg:co:register} Co-registration of pairs of successive fused images (section
  \ref{sec:cli:intraregistration:coregistration}). This yields the
  transformations $T_{t+1 \leftarrow t}$. Fused images are located in
  \verb|<EMBRYO>/FUSE/FUSE_<EXP_FUSE>|: the parameter \verb|EXP_FUSE|
  is either set in the parameter file or is set at
  \verb|RELEASE|. This step may be long. 

\item Composition of transformations issued from the co-registration
  step. This step computes the transformations $T_{ref \leftarrow
    t}$. towards a reference image \verb|ref| given by the parameter
  \verb|intra_registration_reference_index|. 

\item \label{it:intrareg:co:template} Computation of the \textit{template} image (section
  \ref{sec:cli:intraregistration:template}). This \textit{template}
  image dimension are computed so that the useful information of all
  resampled images fits into it. Useful information can be issued from
  either the fused sequence, the segmentation sequence or the
  post-segmentation sequence. It is indicated by the
  \verb|intra_registration_template_type| which value can be either
  \verb|'FUSION'|,  \verb|'SEGMENTATION'|, or
  \verb|'POST-SEGMENTATION'|. This step may be long. 

\item \label{it:intrareg:co:resampling}
  Resampling of either the fused or the segmentation images
  (section \ref{sec:cli:intraregistration:resampling}). Note that
  changing the parameters for this step will not require to re-compute
  the first steps. 

\item \label{it:intrareg:co:movies}
  Extraction of 2D+t images from the resampled sequences (section
  \ref{sec:cli:intraregistration:movies}). Note that changing the
  parameters for this step (i.e. requiring extra movies) will not
  require to re-compute the first steps, with an eventual exception
  for the resampling step. 

\end{enumerate}



\subsection{\texttt{1.5-intraregistration.py} options}

The following options are available:
\begin{description}
  \itemsep -1ex
\item[\texttt{-h}] prints a help message
\item[\texttt{-p \underline{file}}] set the parameter file to be parsed
\item[\texttt{-e \underline{path}}] set the
  \texttt{\underline{path}} to the directory where the
  \texttt{RAWDATA/} directory is located
\item[\texttt{-t \underline{file}}] set the resampling transformation file for the reference image (see section \ref{sec:cli:intraregistration:template})
\item[\texttt{-a \underline{string}}] set the resampling transformation angles for the reference image (see section \ref{sec:cli:intraregistration:template})
\item[\texttt{-k}] allows to keep the temporary files
\item[\texttt{-f}] forces execution, even if (temporary) result files
  are already existing
\item[\texttt{-v}] increases verboseness (both at console and in the
  log file)
\item[\texttt{-nv}] no verboseness
\item[\texttt{-d}]  increases debug information (in the
  log file)
\item[\texttt{-nd}] no debug information
\end{description}


\subsection{Output data}

The results are stored in sub-directories
\texttt{INTRAREG/INTRAREG\_<EXP\_INTRAREG>} under the
\texttt{/path/to/experiment/} directory where \texttt{<EXP\_INTRAREG>} is the value of the variable \texttt{EXP\_INTRAREG} (its
default value is '\texttt{RELEASE}'). 

\mbox{}
\dirtree{%
.1 /path/to/experiment/.
.2 \ldots.
.2 INTRAREG/.
.3 INTRAREG\_<EXP\_INTRAREG>/.
.4 CO-TRSFS/.
.4 [FUSE/].
.4 LOGS/.
.4 [MOVIES/].
.4 [POST/].
.4 [SEG/].
.4 TRSFS\_t<begin>-<end>/.
.2 \ldots.
}
\mbox{}

Output data are of two kinds: image series (fused images, segmentation
images, post-corrected segmentation images) can be resampled in the
same common geometry (also known as the \textit{template}), see
section \ref{sec:cli:intraregistration:resampling}, and 3D (ie 2D+t)
images of the evolution (with respect to time) of one section (XY, XZ,
or YZ) of the images of the series can be built, see
section \ref{sec:cli:intraregistration:movies}.




\subsection{Intra-registration parameters}

\subsubsection{Step \ref{it:intrareg:co:register}: co-registration}
\label{sec:cli:intraregistration:coregistration}
Default registration parameters for the co-registration are set by:
\begin{verbatim}
# intra_registration_compute_registration = True
# intra_registration_transformation_type = 'rigid'
# intra_registration_transformation_estimation_type = 'wlts'
# intra_registration_lts_fraction = 0.55
# intra_registration_pyramid_highest_level = 6
# intra_registration_pyramid_lowest_level = 3
# intra_registration_normalization = True
\end{verbatim}
Computed transformations are stored in \verb|INTRAREG/INTRAREG_<EXP\_INTRAREG>/CO-TRSFS|.
It may be advised to set the pyramid lowest level value to some higher value to speed up the co-registrations (recall that all pairs of successive images will be co-registered, i.e.
\begin{verbatim}
intra_registration_pyramid_lowest_level = 4
\end{verbatim}

Co-registration are computed using the fused images of
\texttt{/path/to/experiment/FUSE/FUSE\_<EXP\_FUSE>}. If
\texttt{EXP\_FUSE} is a list of strings (ie indicates a list a
directories) rather than a single string, the fused image from the
first directory are used for the co-registration computation.

Typically, if there are several fused series (eg, in case of multi-channel
acquisition) as in

\mbox{}
\dirtree{%
.1 /path/to/experiment/.
.2 \ldots.
.2 FUSE/.
.3 FUSE\_MEMBRANES/.
.3 FUSE\_NUCLEI/.
.2 \ldots.
}
\mbox{}

Specifying
\begin{verbatim}
EXP_FUSE = ['MEMBRANES', 'NUCLEI']
\end{verbatim}
in the parameter file implies that co-registrations will be done on
the fused images from \texttt{FUSE/FUSE\_MEMBRANES/}.



\subsubsection{Step \ref{it:intrareg:co:template}: template building}
\label{sec:cli:intraregistration:template}

\begin{verbatim}
# intra_registration_reference_index = None
# intra_registration_reference_resampling_transformation_file = None
# intra_registration_reference_resampling_transformation_angles = None
#
# intra_registration_template_type = "FUSION"
# intra_registration_template_threshold = None
# intra_registration_margin = None
#
# intra_registration_resolution = 0.6
#
# intra_registration_rebuild_template = False
\end{verbatim}

\begin{itemize}
\itemsep -1ex

\item The \verb|intra_registration_reference_index| allows to choose the reference image (the one which remains still, i.e. up to a translation), by default it is the first image image of the series (associated to \verb|begin|). 
However, it may happen that this image has to be reoriented to fit the user's expectation. The resampling transformation\footnote{The resampling transformation is the one that goes from the destination image towards the input image.}, that re-orient the reference image, can then be given and will be applied to the whole series.
\begin{itemize}
\itemsep -1ex
\item \verb|intra_registration_reference_resampling_transformation_file| can be given a resampling transformation file name.
\item \verb|intra_registration_reference_resampling_transformation_angles| can be given a string describing the successive rotations (with respect to the frame axis) to be applied. E.g. the string \verb|"X 30 Y 50"| defines a resampling transformation equal to $R_X(30) \circ R_Y(50)$ where $R_X(30)$ is a rotation of 30 degrees around the X axis and $R_Y(50)$ is a rotation of 50 degrees around the Y axis.
\end{itemize}

\item Depending on \verb|intra_registration_template_type| (\verb|'FUSION'|,
\verb|'SEGMENTATION'| or \verb|'POST-SEGMENTATION'|), the two latter
assume obviously that the segmentation has been done), the
\textit{template} image can be built either after the fusion or the
segmentation images. If no threshold is given by
\verb|intra_registration_template_threshold|, the built template will
be large enough to include all the transformed fields of view (in this
case, the template is the same whatever
\verb|intra_registration_template_type| is).

If \verb|intra_registration_template_type='FUSION'| (respectively
\verb|'SEGMENTATION'| and \verb|'POST-SEGMENTATION'|),  the template
is built from the images of the first directory indicated by
\texttt{EXP\_FUSE} (respectively
\texttt{EXP\_SEG} and \texttt{EXP\_POST}) in case of
\texttt{EXP\_FUSE} contains a list of strings.

If a threshold is given, the built template will be large enough to
include all the transformed points above the threshold. E.g., the
background is labeled with either '1' or '0' in segmentation images,
then a threshold of '2' ensures that all the embryo cells will not be
cut by the resampling stage.  In this case, adding an additional
margin (with \verb|intra_registration_margin|) to the template could be a good idea for visualization
purpose. 

\item Specifying  using a different resolution for the drift-compensated series than the
\verb|target_resolution| (the resolution of the fused images) allows
to decrease the resampled images volume. This can be achieved by
setting \verb|intra_registration_resolution| to the desired  value  (default is 0.6).

\item 
Last, co-registrations may have been computed during a first
computation, fused images being used to compute the template. However,
if  a subsequent segmentation has been conducted, a smaller template
is likely to be computed (with the segmentation images to build the
template), without recomputing the co-registration. This is the
purpose of the variable
\texttt{intra\_registration\_rebuild\_template}.
If set to \texttt{True}, it forces to recompute the template as well
as the transformations from the co-registrations (that are not
re-computed). Obviously, resampling as well as 2D+t movies are also
re-generated.
\end{itemize}

As an example, building a \textit{template} image after the segmentation images can be done with
\begin{verbatim}
# intra_registration_reference_index = None
intra_registration_template_type = "SEGMENTATION"
intra_registration_template_threshold = 2
# intra_registration_resolution = 0.6
intra_registration_margin = 10
\end{verbatim}


Computed transformations from the \textit{template} image as well as the \textit{template} image itself are stored in \verb|INTRAREG/INTRAREG<EXP_INTRAREG>/TRSFS_t<F>-<L>/| where \verb|<F>| and \verb|L| are the first and the last index of the series (specified by \verb|begin| and \verb|end| from the parameter file).


\subsubsection{Step \ref{it:intrareg:co:resampling}: resampling fusion/segmentation images}
\label{sec:cli:intraregistration:resampling}

The resampling of the fused and/or segmentation images are done
depending on the value of the following variables (here commented). Resampling is done
either if the following parameters are set to \verb|True| or if movies
are requested to be computed (section \ref{sec:cli:intraregistration:movies}).
\begin{verbatim}
# intra_registration_resample_fusion_images = True
# intra_registration_resample_segmentation_images = False
# intra_registration_resample_post_segmentation_images = False
\end{verbatim}
This default behavior implies that the fusion images will be resampled
while the segmentation and the post-corrected segmentation images are not.


Resampled images will be stored in the
\texttt{INTRAREG/INTRAREG\_<EXP\_INTRAREG/>} directory, with the same
hierarchy than under \texttt{/path/to/experiment}. E.g. 

\mbox{}
\dirtree{%
.1 /path/to/experiment/.
.2 \ldots.
.2 FUSE/.
.3 FUSE\_1/.
.3 FUSE\_2/.
.2 \ldots.
}
\mbox{}

Specifying
\begin{verbatim}
EXP_FUSE = ['1', '2']
\end{verbatim}
in the parameter file causes the resampling of both fused image series
(\texttt{FUSE/FUSE\_1/} and \texttt{FUSE/FUSE\_2/})

\mbox{}
\dirtree{%
.1 /path/to/experiment/.
.2 \ldots.
.2 FUSE/.
.3 FUSE\_1/.
.3 FUSE\_2/.
.2 INTRAREG/.
.3 INTRAREG\_<EXP\_INTRAREG>/.
.4 FUSE/.
.5 FUSE\_1/.
.5 FUSE\_2/.
.4 \ldots.
.2 \ldots.
}
\mbox{}

The same behavior stands for \texttt{EXP\_SEG} and  \texttt{EXP\_POST}.


\subsubsection{Step \ref{it:intrareg:co:movies}: 2D+t movies}
\label{sec:cli:intraregistration:movies}
For either visual assessment or illustration purposes, 2D+t (i.e. 3D) images can be built from 2D sections extracted from the resampled temporal series. This is controlled by the following parameters:
\begin{verbatim}
# intra_registration_movie_fusion_images = True
# intra_registration_movie_segmentation_images = False
# intra_registration_movie_post_segmentation_images = False

# intra_registration_xy_movie_fusion_images = [];
# intra_registration_xz_movie_fusion_images = [];
# intra_registration_yz_movie_fusion_images = [];

# intra_registration_xy_movie_segmentation_images = [];
# intra_registration_xz_movie_segmentation_images = [];
# intra_registration_yz_movie_segmentation_images = [];

# intra_registration_xy_movie_post_segmentation_images = [];
# intra_registration_xz_movie_post_segmentation_images = [];
# intra_registration_yz_movie_post_segmentation_images = [];
\end{verbatim}

If \verb|intra_registration_movie_fusion_images| is set to \verb|True|, a movie is made with the  XY-section located at the middle of each resampled fusion image (recall that, after resampling, all images have the same geometry). Additional XY-movies can be done by specifying the wanted Z values in \verb|intra_registration_xy_movie_fusion_images|. E.g.
\begin{verbatim}
intra_registration_xy_movie_fusion_images = [100, 200];
\end{verbatim}
will build two movies with XY-sections located respectively at Z values of 100 and 200. The same stands for the other orientation and for the resampled segmentation images.

Movies will be stored in the
\texttt{INTRAREG/INTRAREG\_<EXP\_INTRAREG>/MOVIES/} directory, with the same
hierarchy than under \texttt{/path/to/experiment}. E.g., 
\begin{verbatim}
EXP_FUSE = ['1', '2']
\end{verbatim}
in the parameter file results in

\mbox{}
\dirtree{%
.1 /path/to/experiment/.
.2 \ldots.
.2 FUSE/.
.3 FUSE\_1/.
.3 FUSE\_2/.
.2 INTRAREG/.
.3 INTRAREG\_<EXP\_INTRAREG>/.
.4 FUSE/.
.5 FUSE\_1/.
.5 FUSE\_2/.
.4 MOVIES/.
.5 FUSE/.
.6 FUSE\_1/.
.6 FUSE\_2/.
.4 \ldots.
.2 \ldots.
}
\mbox{}

The same behavior stands for \texttt{EXP\_SEG} and  \texttt{EXP\_POST}.

\subsection{Parameter list}

Please also refer to the file
\texttt{parameter-file-examples/1.5-intraregistration-parameters.py}

\begin{itemize}
\itemsep -1ex
\item \texttt{EN}
\item \texttt{EXP\_FUSE} \\
  String (\texttt{str} type) or list (\texttt{list} type) of
  strings. It indicates what are the fused images directories, of the form
  \texttt{/path/to/experiment/FUSE/FUSE\_<EXP\_FUSE>}.
\begin{verbatim}
EXP_FUSE = 'exp1'
EXP_FUSE = ['exp1', 'exp2']
\end{verbatim}
  are both valid. Recall that the default value of \texttt{EXP\_FUSE}
  is \texttt{'RELEASE'}.
\item \texttt{EXP\_INTRAREG}
\item \texttt{EXP\_POST}
\item \texttt{EXP\_SEG}
\item \texttt{PATH\_EMBRYO}
\item \texttt{begin}
\item \texttt{default\_image\_suffix}
\item \texttt{delta}
\item \texttt{end}
\item \texttt{intra\_registration\_compute\_registration}
\item \texttt{intra\_registration\_lts\_fraction}
\item \texttt{intra\_registration\_margin}
\item \texttt{intra\_registration\_movie\_fusion\_images}
\item \texttt{intra\_registration\_movie\_post\_segmentation\_images}
\item \texttt{intra\_registration\_movie\_segmentation\_images}
\item \texttt{intra\_registration\_normalization}
\item \texttt{intra\_registration\_pyramid\_highest\_level}
\item \texttt{intra\_registration\_pyramid\_lowest\_level}
\item \texttt{intra\_registration\_rebuild\_template}\\
  If set to \texttt{True}, force to recompute the template as well as the
  transformations from existing co-registrations (that are not
  re-computed). It is useful when a first intra-registration has been
  done with only the fusion images: a second intra-registration with
  the segmentation images as template can be done without recomputing the co-registrations.
\item \texttt{intra\_registration\_reference\_index}
\item \texttt{intra\_registration\_reference\_resampling\_transformation\_file}
\item \texttt{intra\_registration\_reference\_resampling\_transformation\_angles}
\item \texttt{intra\_registration\_resample\_fusion\_images}
\item \texttt{intra\_registration\_resample\_post\_segmentation\_images}
\item \texttt{intra\_registration\_resample\_segmentation\_images}
\item \texttt{intra\_registration\_resolution}
\item \texttt{intra\_registration\_sigma\_segmentation\_images}
\item \texttt{intra\_registration\_template\_threshold}
\item \texttt{intra\_registration\_template\_type}
\item \texttt{intra\_registration\_transformation\_estimation\_type}
\item \texttt{intra\_registration\_transformation\_type}
\item \texttt{intra\_registration\_xy\_movie\_fusion\_images}
\item \texttt{intra\_registration\_xy\_movie\_post\_segmentation\_images}
\item \texttt{intra\_registration\_xy\_movie\_segmentation\_images}
\item \texttt{intra\_registration\_xz\_movie\_fusion\_images}
\item \texttt{intra\_registration\_xz\_movie\_post\_segmentation\_images}
\item \texttt{intra\_registration\_xz\_movie\_segmentation\_images}
\item \texttt{intra\_registration\_yz\_movie\_fusion\_images}
\item \texttt{intra\_registration\_yz\_movie\_post\_segmentation\_images}
\item \texttt{intra\_registration\_yz\_movie\_segmentation\_images}
\item \texttt{result\_image\_suffix}
\end{itemize}
