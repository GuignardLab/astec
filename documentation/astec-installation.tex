
\chapter{Installation}

ASTEC (acronym of ``adaptive segmentation and tracking of embryonic
cells'' \cite{guignard:tel-01278725}) has been designed for unix-like
systems (e.g. Linux, or MacOS).
%
It has been developed with \texttt{python2.7} and was not tested for
\texttt{python3.0}.
%
It is a set of \texttt{python} scripts, built over a set of \texttt{C} commands.


There are two distributions. The first one can be retrieved from
\url{github.com/astec-segmentation/astec-2019-published} and includes
both the \texttt{python} and the \texttt{C} codes. The installation
procedure is dedicated to this distribution.

The second one is devoluted to more advanced users that may want to
benefit from future developements of the ASTEC distribution:
\begin{itemize}
\itemsep -0.5ex
\item \texttt{python} scripts can be retrieved from
  \url{github.com/astec-segmentation/astec}
\item \texttt{C} code can be retrieved from
  \url{gitlab.inria.fr/morpheme/vt}
\item optional third-party librairies can be retrieved from  \url{gitlab.inria.fr/morpheme/vt-third-party}
\end{itemize}  

Both \url{github.com/astec-segmentation/astec-2019-published} and
\url{github.com/astec-segmentation/astec} contains the following 4
sub-directories

\mbox{}
\dirtree{%
.1 astec[-2019-published/].
.2  documentation/.
.2 parameter-file-examples/.
.2 src/.
.2 tutorial/.
}
\mbox{}

\begin{itemize}
\itemsep -0.5ex
\item \texttt{documentation/} contains this documentation.
\item \texttt{parameter-file-examples/} contains templates of
  parameter files for the \texttt{python} scripts. See chapter
  \ref{chap:clis} for further details.
\item \texttt{src/} contains the \texttt{python} scripts and files (as
  well as the \texttt{C} codes for the \texttt{astec-2019-published} distribution).
\item \texttt{tutorial/} contains a toy data set and the associated
  parameter files. See chapter \ref{chap:tutorial}.
\end{itemize}  


\section{\texttt{github.com/astec-segmentation/astec}}

\subsection{Linux system}

This section describes the required command to install the ASTEC
distribution on a Linux system (was tested on a Ubuntu system
(18.04.2, 64 bits) installed on a virtual
machine\footnote{\url{virtualbox.org}}) so the tutorial (chapter \ref{chap:tutorial}) can be run.

\begin{enumerate}
% \itemsep -0.5ex
\item Get the distribution. It is recommended (but not necessary) to
  use \texttt{git}, so keeping up to date with the distribution will
  be easier.   \texttt{git} can be installed with
  \begin{code}{0.8}
  \$ sudo apt install git
  \end{code}
  Then, choose the directory where to install the ASTEC distribution,
  and download it
  \begin{code}{0.8}
    \$ cd /wherever/one/wants/ \\
    \$ git clone https://github.com/astec-segmentation/astec-2019-published.git
  \end{code}
  It creates the directory
  \texttt{/wherever/one/wants/astec-2019-published/} that will be
  denoted \texttt{/path/to/astec/} from now on.


\item Prepare the compilation of the \texttt{C} code. Compilation is done within the
  \texttt{cmake}\footnote{\url{cmake.org}} framework. The standard
  Ubuntu distribution comes with a \texttt{C} compiler but not with a
  \texttt{C++} one. Last a development version of the \texttt{zlib} is
  required. The next few lines allow to install the required components.
  \begin{code}{0.8}
    \$ sudo apt install cmake \\
    \$ sudo apt install cmake-curses-gui \\
    \$ sudo apt install g++ \\
    \$ sudo apt install zlib1g-dev
  \end{code}
\item Compile the \texttt{C} code.
    \begin{code}{0.8}
    \$ cd /path/to/astec/ \\
    \$ cd src/ASTEC/CommunFunctions/cpp/vt/ \\
    \$ mkdir build \\
    \$ cd build \\
    \$ cmake ../ \\
    \$ make \\
  \end{code}

\item Install the required \texttt{python} libraries. As mentioned,
  ASTEC has been developed with \texttt{python2.7}. \texttt{pip} is
  here used for the installation of the \texttt{python}
  libraries. Required libraries are \texttt{numpy}, \texttt{scipy},
  \texttt{libtiff}, and \texttt{h5py}.
  \begin{code}{0.8}
    \$ sudo apt install python2.7 \\
    \$ sudo apt install python-pip \\
    \$ sudo pip install numpy \\
    \$ sudo pip install scipy \\
    \$ sudo pip install libtiff \\
    \$ sudo pip install h5py \\
  \end{code}

\item Make the ASTEC scripts/commands available as on-line commands.
It can be done in a terminal (but will be valid only for this terminal)
\begin{code}{0.8}
  \$ export PATH=\${PATH}:/path/to/astec/src
\end{code}
or by adding the above line in the right setup file (e.g. \texttt{.bashrc}, \texttt{.profile}, \ldots).

\end{enumerate}









