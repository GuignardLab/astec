\section{\texttt{3-manualcorrection.py}}
\label{sec:cli:manual:correction}

The seeded watershed is likely to produce segmentation errors, even with a careful choice of parameters. It is advised to set the parameters to favour over-segmentations insted of under-segmentations since the former are much more easier to correct, which is the purpose of \texttt{3-manualcorrection.py}.


\subsection{\texttt{3-manualcorrection.py} options}

The following options are available:
\begin{description}
 \itemsep -0.5ex
\item[\texttt{-h}] prints a help message
\item[\texttt{-p \underline{file}}] indicates the parameter file to be parsed
\item[\texttt{-e \underline{path}}] indicates the
  \texttt{\underline{path}} to the directory where the
  \texttt{RAWDATA/} directory is located
\item[\texttt{-k}] allows to keep the temporary files
\item[\texttt{-f}] forces execution, even if (temporary) result files
  are already existing
\item[\texttt{-v}] increases verboseness (both at console and in the
  log file)
\item[\texttt{-nv}] no verboseness
\item[\texttt{-d}]  increases debug information (in the
  log file)
\item[\texttt{-nd}] no debug information
\item[\texttt{-i \underline{input\_image}}] set the \texttt{\underline{input\_image}} file to be corrected. Allows to skip the automated naming of files.
  \item[\texttt{-o \underline{output\_image}}] set the resulting \texttt{\underline{ouput\_image}} file to be saved. Allows to skip the automated naming of files. 
\item[\texttt{-m \underline{mapping\_file}}] set the \texttt{\underline{mapping\_file}} to be used for the correction.
\item[\texttt{-nsc \underline{smallest\_cells}}] set the number of the smallest cells to be displayed after correction. The smallest cells are the most likely to be issued from an over-segmentation.
\item[\texttt{-nlc \underline{largest\_cells}}] set the number of the largest cells to be displayed after correction. The largest cells are the most likely to be issued from an under-segmentation.  
\end{description}



\subsection{Output data}

The results are stored in sub-directories
\texttt{SEG/SEG\_<EXP\_SEG>} under the
\texttt{/path/to/experiment/} directory where \texttt{<EXP\_SEG>} is the value of the variable \texttt{EXP\_SEG} (its
default value is '\texttt{RELEASE}').
\texttt{<EN>\_seg\_t<begin>.inr} is the correction of the segmentation image \texttt{<EN>\_mars\_t<begin>.inr}.

\dirtree{%
.1 /path/to/experiment/.
.2 \ldots.
.2 SEG/.
.3 SEG\_<EXP\_SEG>/.
.4 <EN>\_mars\_t<begin>.inr.
.4 <EN>\_seg\_t<begin>.inr.
.4 LOGS/.
.4 RECONSTRUCTION/.
.2 \ldots.
}

\subsection{Segmentation correction parameters}

\texttt{3-manualcorrection.py} parses a correction file whose name is given by the variable \texttt{mancor\_mapping\_file}. The syntax of this file is very simple. Lines beginning with \texttt{\#} are ignored (and can be used to insert comments in the files). Non-empty lines should contain two numbers separated by a space, and \texttt{3-manualcorrection.py} will replace the first number by the second in the segmentation file.

E.g. a cell $c$ is recognized to be over-segmented, and then is represented by two labels, says 9 and 10. Thus the line
\begin{framed}
\begin{verbatim}
10 9
\end{verbatim}
\end{framed}
will replace all 10's by 9's in the segmentation image,  thus $c$ will only be represented by 9's after correction. See also the tutorial section \ref{sec:tutorial:manual:correction} for an other example.

\subsection{Parameter list}

Please also refer to the file
\texttt{parameter-file-examples/3-manualcorrection-parameters.py}

\begin{itemize}
\itemsep -0.5ex
\item \texttt{EN}
\item \texttt{EXP\_SEG}
\item \texttt{PATH\_EMBRYO}
\item \texttt{begin}
\item \texttt{default\_image\_suffix}
\item \texttt{delta}
\item \texttt{mancor\_input\_seg\_file}
\item \texttt{mancor\_mapping\_file}
\item \texttt{mancor\_output\_seg\_file}
\item \texttt{mars\_begin}
\item \texttt{mars\_end}
\item \texttt{result\_image\_suffix}
\end{itemize}
